\section{Methodology}
%requires checking
To determine if {\it de Volkskrant} is politically biased or not, we first need to determine what the tone of an article is. This is done based on the words in the article. The words will be either negative, positive or neutral based on the words listed in our data set. So if most of the words are negative then the article is probably written from a disagreeing perspective. This can also be said for positive words. \\

We are using the following data set\cite{sub:words} to classify the words in an article. These words in this data set have a polarity score between +1.0 and -1.0 and  these values stand for respectively positive or negative. The data set also contains a subjectivity score, which is used to display if the word represents a fact or an opinion and it ranges from 0.0 to +1.0. \\

In order to test the hypothesis we start by processing the HTML data in every single article to obtain information about the author and the description (the first paragraph, usually bold in newspapers), which are not present in the data set. Then use the data set with the additional information from the previous step to calculate polarity and subjectivity on article description and text using the a natural language processing tool for sentiment analysis in Dutch language\cite{tool:pattern.nl}. \\

De Smedt and Daelemans \cite{validate:tool} validated the natural language processing tool for sentiment analysis in Dutch language \cite{tool:pattern.nl}. They tested the tool with a set of 2,000 Dutch book reviews, which resulted in 82\% accuracy (P 0.79, R 0.86).  \\

Firstly, the overall sentiment of the newspaper to define a baseline for comparison by determinng if it is generally positive or negative and how objective it is. Later, assuming that all relevant articles are in the category "Politiek", we filter out all other articles that are not in the fore mentioned category. An additional filtering is applied to sieve the articles based on mentions of a specific political party using regular expressions. The text matching rules is derived from the list parliamentary parties part of the House of Representatives\cite{web:parliamentary-parties} using their full names and abbreviations. The articles are filtered in inclusive and exclusive way, as for the former counts an article to a political party if the party name or abbreviation is matched in the text. However, the exclusive filtering counts an article to a political party only if no other political party is mentioned in the same article.

The filtered data is further fed to the computation cluster to generate the average polarity and subjectivity scores grouped by political party, year, author or combination of these. The results is the overall tone of articles related to political party, in general or in the context of an author or а year. The chronological element of the data will allows us to observe the change of polarity and subjectivity towards different political parties over time. The resulting scores are the base used to determine whether {\it de Volkskrant} is politically biased and which political parties are favoured or disfavoured by the press. \\

Finally, when all results are obtained, the results will be validated using different data sources. To test whether the obtained results are valid, a one-sample{\it Z}-test will be performed. Using this {\it Z}-test, there will be checked if the mean of the polarity of political parties differ significantly from the overall mean. The same {\it Z}-test will also be performed for the subjectivity scores of political parties.There are two kinds of hypotheses for a one sample {\it Z}-test, the null hypothesis and the alternative hypothesis. The alternative hypothesis assumes that some difference exists between the true mean and the and the comparison value, whereas the null hypothesis assumes that no difference exists.