\section{Conclusion}
Based on our results there can be concluded that the Volkskrant has a bias towards several political parties. There can be concluded that {\it de Volkskrant} is left motivated. The political bias of {\it de Volkskrant} is also shown by the fact that articles about different parties vary a lot in subjectivity. Based on these results the hypothesis described in the Introduction will be rejected, and the Null hypothesis will be accepted. \\

While the results of the performed research are very outspoken, some improvements can still be made by future researchers in this field. One important thing which is noticed is that {\it de Volkskrant} has a payment plan that blocks parts of the article if you don't have an subscription. This makes the calculations done on the article text less representative to the actual values. We would like to suggest switching or adding to the data set using a different newspaper or news site like 'nu.nl' with all their articles fully available. This would result in a better analysis of the tones of political motivated articles. \\

Another improvement would be to test the polarity and subjectivity scores of more parties. Although the polarity and subjectivity scores have been calculated for every party, {\it Z}-tests could not be performed for all parties.