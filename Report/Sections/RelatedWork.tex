\section{Related Work}
Study on Political Views of the Internet Recent research has increasingly investigated the political views of the Internet. The research can be viewed according to the media they deal with, i.e., social media and traditional news media. Many researchers have focused their research on social media, such as blogs\cite{research:blogpost}. The political views are expressed freely and explicitly in social media. The people that are using social media are not restrained by journalism values such as fairness or balance, and do not go through a formal editorial process. Different from these works, our work studies the political views of the traditional mainstream media. \\

As researched by Yoonhyeung Choi and Ying-Hsuan Lin\cite{hurricanes:newspaper}, newspapers are required to inform the general public. They have researched how newspapers structure their articles before a hurricane hits populated areas. And they have concluded that the articles beforehand are not taking in account what the predicited outcome is of preventive action. They also concluded that the newspapers should use a more emotional frame than a logical frame for their articles.
However we are not going to see how well the newspapers are informing people, but if they write colored articles. \\

It becomes more difficult to identify political views from traditional news media contents. Normally they are not free from political bias\cite{bias:newspaper}, however the news producers usually do not explicitly present their political views in the news. The most news articles mainly deliver facts than opinions. However there are some types of articles that express opinions, examples are editorials or columns. The political views can be also expressed in many different ways\cite{book:mediabias}: through fact selection, by omitting detailed facts or selecting information sources; through writing style, by choice of words and tone; through presentation style, e.g., selection of photos. \\

Sentiment analysis of natural language texts is a large and growing field. Pang et al. \cite{sentiment:machinelearning} perform sentiment analysis of movie reviews. Their results show that the machine learning techniques perform better than simple counting methods. Due to the restricted time for the research we were not able to use a machine learning tool. In \cite{sentiment:sentences}, they identify which sentences in a review are of subjective character to improve sentiment analysis. Subjectivity analysis refers to analysis of language used in expressing opinions, evaluation, and emotions. In our research we making the same distinction between the sentiment of a news article and the subjectivity of a news article. 